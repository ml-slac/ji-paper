\documentclass{article}
\usepackage{jheppub}
\usepackage{amsmath}
\usepackage{graphicx}
\usepackage{amssymb}
\usepackage{subfig}
\usepackage{natbib}
\bibliographystyle{JHEP-2}
\usepackage[utf8]{inputenc}

\title{Jet-Images -- Deep Learning Edition}
\author{Luke de Oliveira,${}^a$}
\author{Michael Kagan,${}^{b}$}
\author{Lester Mackey,${}^c$}
\author{Benjamin Nachman,${}^{b}$ and}
\author{Ariel Schwartzman${}^b$}

\affiliation{$^{a}$ Institute for Computational and Mathematical Engineering, Stanford University, Stanford, CA 94305, USA}

\affiliation{$^{b}$SLAC National Accelerator Laboratory, Stanford University, 2575 Sand Hill Rd, Menlo Park,
  CA 94025, U.S.A.}

\affiliation{$^{a}$Department of Statistics, Stanford University, Stanford, CA 94305, USA}

\emailAdd{lukedeo@stanford.edu, mkatan@cern.ch, lmackey@stanford.edu, bnachman@cern.ch, sch@slac.stanford.edu}

\abstract{Building on the notion of a particle physics detector as a camera and the collimated streams of high energy particles it measures as an image, we investigate the potential of machine learning techniques based on deep learning architectures.  Modern deep learning algorithms trained on {\it jet images} can out-perform standard physically-motivated feature driven approaches to jet tagging.  We develop techniques for visualizing where these features are learned by the network and what additional information is used to improve performance. This feedback loop between physically-motivated feature driven tools and unsupervised learning algorithms is general and can be used to significantly increase the sensitivity to discover new particles and new forces.}

\begin{document}

\maketitle

\section{Introduction}

The fundamental challenge at the energy frontier of particle physics is identifying subtle signals beneath enormous backgrounds. A monumental triumph was the recent discovery of the Higgs boson - a particle that is produced one in every $10^{10}$ collisions at the Large Hadron Collider (LHC). Machine learning techniques have played a key role in all aspects of this search.  At the most basic level, charged particle tracks are reconstructed using pattern recognition techniques, higher level objects, blah blah. 

Techniques built on physically motivated features have succesfully probed the highest energies and smallest scales ever achieved in terestrial experiments.  At the same time, there have been significant gains 

The high particle multiplicity final states of hadron collider events 
Something like "we have figured out many cool variables" "machine learning can find variables"  "lets use one to inform/improve the other!"

Places where ML is used in HEP: TMVA~\cite{Hocker:2007ht}

b-tagging: MV1 (ATLAS) and CSV (CMS)
tau-tagging
NP searches
rare SM measurements (e.g. single top)
Higgs

Jet Images paper: http://arxiv.org/abs/1407.5675


Let's keep a running document with all pub-ready plots in here.


We can even use this later on when we actually write the paper -- this hosting allows templates, styles, etc. and we can all edit, which is great.

If you guys have any tips on plotting style, let's keep that here too -- as I make graphics now, I want them to be pub ready. You definitely know what reviewers go for more than I do.

\section{Simulation Details and the Jet Image}
\label{sec:simulation}

In order to study jet images in a realistic scenario, we use Monte Carlo (MC) simulations of high energy particle collisions. One important jet tagging application is the identification of highly Lorentz boosted $W$ bosons decaying into quarks amidst a large background from the generic production of quarks and gluons.  This classification task has been thoroughly studied experimentally\footnote{There is also an extensive literature on phenomenological studies - see references within the experimental papers.}~\cite{Khachatryan:2014vla,ATL-PHYS-PUB-2015-033,ATL-PHYS-PUB-2014-004} and used in many analyses~\cite{Aad:2015owa,Khachatryan:2014hpa,Khachatryan:2015mta,Khachatryan:2015oba,Khachatryan:2015gza,Khachatryan:2015bma,Khachatryan:2015cwa,Khachatryan:2015ywa,Aad:2014wea,Aad:2015agg,Aad:2015kna,Aad:2015ufa,Aad:2014haa}.  

To simulate highly boosted $W$ bosons, a hypothetical $W'$ boson is generated and forced to decay to a hadronically decaying $W$ boson ($W\rightarrow qq'$) and a $Z$ boson which decays invisibly ($Z\rightarrow \nu\bar{\nu}$).  The mass of the $W'$ boson determines the Lorentz boost of the $W$ boson in the lab frame since the $W'$ is produced nearly at rest and the $W$ boson momentum is approximately $m_{W'}/2$.  The invisible decay of the $Z$ boson ensures that the jet in the event with the highest transverse momentum is the $W$ boson jet.  Multijet production of quarks and gluons is simulated as a background.  Both the $W'$ signal and the multijet background are generated using \textsc{Pythia} 8.170~\cite{Pythia8,Pythia} at $\sqrt{s}=14$ TeV.  The angular separation of the $W$ boson decay products in the plane transverse to the beam direction scales as $2m_{W}/p_{T,W}$, where $m_W\approx 80$~GeV and $p_{T,W}$ is the component of the $W$ boson momentum in this plane.  The tagging strategy and performance depend strongly on $p_{T,W}$, so we focus on a particular range: $250$~GeV~$<p_{T,W}<300$~GeV.  This corresponds to an angular spread of about $1$ radian.  The decay products of the $W$ bosons as well as the background are clustered into jets using the anti-$k_t$ algorithm~\cite{antiktpaper} via \textsc{FastJet}~\cite{fastjet} 3.0.3.  To mitigate the contribution from the underlying event, jets are are trimmed~\cite{trimming} by re-clustering the constituents into $R=0.3$ $k_t$ subjets and dropping those which have $p_T^\text{subjet}<0.05\times p_T^\text{jet}$. 

To model the discretization and finite acceptance of a real detector, a calorimeter of towers with size $0.1\times 0.1$ in $(\eta,\phi)$ extends out to $\eta=5.0$.  The total energy of the simulated particles incident upon a particular cell are added as scalars and the four-vector $p_j$ of any particular tower $j$ is given by

\begin{align}
\label{eq:calo}
p_j = \sum_{i\text{ incident on $j$}}E_i(\cos\phi_j/\cosh \eta_j,\sin\phi_j/\cosh \eta_j,\sinh \eta_j/\cosh \eta_j,1),
\end{align}

\noindent where $E_i$ is the energy of particle $i$ and the center of the tower $j$ is $(\eta_j,\phi_j)$.  Towers are treated as massless.

 A {\it jet image} is formed by taking the constituents of a jet and discretizing its energy into pixels in ($\eta,\phi$), with the intensity of each pixel given by the sum of the energy of all constituents of the jet inside that ($\eta,\phi$) pixel.  In our studies, we take the jet image pixelation to match the simulated calorimeter tower granularity.  In the next section, we will discuss the nuances of standardizing the coordinates of a jet image as a pre-processing step prior to applying machine learning.  
 
Show some plots of pT, etc?

\section{Pre-processing and the Symmetries of Space-time}

In order for the machine learning algorithms to most efficiency learn discriminating features between signal and background and not learn the symmetries of space-time, the jet images are pre-processed.  This procedure can greatly improve performance and reduce the required size of the sample used for testing.  Our pre-processing procedure happens in four steps: translation, rotation, re-pixelation, and inversion.  To begin, the jet images are translated so that the leading subjet is at $(\eta,phi)=(0,0)$.  Translations in $\phi$ are rotations around the $z$-axis and so the pixel intensity is unchanged by this operation.  On the other hand, translations in $\eta$ are {\it Lorentz boosts} along $z$, which do not preserve the pixel intensity.  Therefore, a proper translation in $\eta$ would modify the pixel intensity.  One simple modification of the jet image to circumvent this change is to replace the pixel intensity $E_i$ with $p_{T,i}=E_i/\cosh(\eta_i)$.  This new definition of intensity is invariant under translations in $\eta$ and is used exclusively for the rest of this paper.

The second step of pre-processing is to rotate the images around the center of the jet.  If a jet has a second subjet, then the rotation is performed so that the second subjet is at $-\pi/2$.  If no second subjet exists, then the jet image is rotated so that the first principle component of the pixel intensity distribution is at $-\pi/2$.  Unless the rotation is by an integer multiple of $\pi/4$, the rotated grid will not line up with the original grid.  Therefore, the energy in the rotated grid must be re-distributed amongst the pixels of the original image grid.  A cublic spline interpolation is used in this case - see Ref.~\cite{Cogan:2014oua} for details.  The last step is a parity flip so that the right side of the jet image has the highest sum pixel intensity.  

Figure~\ref{fig:preprocess} shows the average jet image for $W$ boson jets and QCD jets before and after the rotation, re-pixelation, and inversion steps of the pre-processing.  The more pronounced second-subjet is already pronounced in the left plots of Fig.~\ref{fig:preprocess}, where there is a clear annulus for the signal $W$ jets which is nearly absent for the background QCD jets.  However, after the rotation, the second core of energy is well isolated and localized in the images.  The spread of energy around the leading subjet is more diffuse for the QCD background which consists largely of gluon jets which have an octet radiation pattern compared to the singlet radiation pattern of the $W$ jets, where the radiation is mostly restricted to the region between the two hard cores.

\begin{figure}[bt]
  \begin{center}
        \includegraphics[width=0.99\textwidth]{figures/Image_mass_average_fixed_nonorm.pdf}
      \caption{ The average jet image for signal $W$ jets (top) and background QCD jets (bottom) before (left) and after (right) applying the rotation, re-pixelation, and inversion steps of the pre-processing.  The average is taken over images of jets with $240$ GeV $<p_T<$ 260 GeV and 65~GeV~$<$ mass $<$~95~GeV.
      \label{fig:preprocess} }
    \end{center}
\end{figure}

One standard pre-processing step that is often additionally applied in machine learning is normalization.  A common normalization scheme is the $L^2$ norm such that $\sum I_i^2=1$ where $I_i$ is the intensity of pixel $i$.  This is particularly useful for the jet images where pixel intensities can span many orders of magnitude.  The jet transverse momenta are all around 250 GeV, but this can be spread amongst many pixels or concentrated in only a few and the $L^2$ norm helps mitigate the spread and thus makes training easier for the machine learning algorithm.  However, normalization can distort information contained within the jet image.  Some information, such as the Euclidean distance $\Delta R$ between subjets in $(\eta,\phi)$ is invariant under all of the pre-processing steps as well as normalization.  However, consider the {\it image mass}, 

\begin{align}
m_I^2=\sum_{i<j} E_iE_j(1-\cos(\theta_{ij})),
\end{align}

\noindent where $E_i=I_i/cosh(\eta_i)$ for pixel intensity $I_i$ and $\theta_{ij}$ is the angle between massless four-vectors with $\eta$ and $\phi$ at the $i$ and $j$ pixel centers.  The image mass is not invariant under all pre-processing steps, but does encode useful discrimination information.  As discussed earlier, with the proper choice of pixel intensity, translations preserve the image mass since it is a Lorentz invariant quantity.  However, the rotation pre-processing step does not preserve the image mass.

\begin{figure}[bt]
  \begin{center}
        \includegraphics[width=0.5\textwidth]{figures/ImageMass_Comparison.pdf}\includegraphics[width=0.5\textwidth]{figures/ImageMass_Comparison_back.pdf}
      \caption{ 
      \label{fig:preprocess2} }
    \end{center}
\end{figure}


\begin{figure}[bt]
  \begin{center}
        \includegraphics[width=0.99\textwidth]{figures/ROCs.pdf}
      \caption{ 
      \label{fig:preprocess3} }
    \end{center}
\end{figure}

\clearpage
\newpage

\begin{figure}[bt]
  \begin{center}
      \subfloat[Unweighted $p_T$ distribution \label{subfig:unweighted_pt}]{
        \includegraphics[width=0.5\textwidth]{figures/unweighted-pt-distribution-[250-300].pdf}
      }
      \subfloat[Weighted $p_T$ distribution \label{subfig:weighted_pt}]{
        \includegraphics[width=0.5\textwidth]{figures/weighted-pt-distribution[250-300].pdf}
      }
      \caption{ Jets originating from the $W'\rightarrow WZ$ decay are re-weighted such that their $p_T$ spectrum matches that of QCD jets\label{fig:pt} }
    \end{center}
\end{figure}


\begin{figure}[bt]
  \begin{center}
  
  
      \subfloat[Weighted jet mass distribution \label{subfig:weighted_mass}]{
        \includegraphics[width=0.5\textwidth]{figures/weighted-mass-distribution[250-300].pdf}
      }
      \subfloat[Weighted $\tau_{21}$ distribution \label{subfig:weighted_nsj}]{
        \includegraphics[width=0.5\textwidth]{figures/weighted-tau21-distribution[250-300].pdf}
      }
      \caption{Weighted mass (left) and $n$-subjettiness (right) of samples, with $W'\rightarrow WZ$ decays in red and QCD jets in blue.\label{fig:mass_nsj_spectrum} }
    \end{center}
\end{figure}  



\begin{figure}[bt]
  \begin{center}
  
      \subfloat[Average weighted $W'\rightarrow WZ$ image \label{subfig:weighted_sig}]{
        \includegraphics[width=0.5\textwidth]{figures/sig-im.pdf}
      }
      \subfloat[Average weighted QCD image \label{subfig:weighted_bkg}]{
        \includegraphics[width=0.5\textwidth]{figures/bkg-im.pdf}
      }
      \caption{Weighted $W'\rightarrow WZ$ (left) and QCD (right) average jet-image
      \label{fig:meanImages} }
    \end{center}
\end{figure}  

Discuss the creation of the jet images.

Discuss the physical differences between W bosons and q/g jets?


\section{Figure of Merit} % (fold)
\label{sec:figure_of_merit}

As is commonly done in High Energy Physics, we eschew the commonly chosen metric of basic accuracy in favor of the Receiver Operating characteristic. This is because we must examine the entire spectrum of trade-off between Type-I and Type-II error, as many applications in physics will choose different points along the trade-off curve. We use a slight modification of the traditional ROC. For any discriminating variable, let $c$ be a threshold on the likelihood ratio on that variable, and let $w$ be the vector of weights over the entire evaluation sample. We define the \emph{rejection} of such a threshold is defined as 
$$
    \rho(c) = \frac{1}{\text{FPR}(c, w)},
$$
where $\text{FPR}(c, w)$ is the weighted false positive rate for using $c$ as a threshold.

We define the \emph{efficiency} of $c$ as 
$$
    \varepsilon(c) = \text{TPR}(c, w),
$$
where $\text{TPR}(c, w)$ is the weighted false positive rate for using $c$ as a threshold. We then evaluate our algorithms using the area under the line generated by $\{(\varepsilon(c), \rho(c)) : \varepsilon(c)\in [0.2, 0.8]\}$. We say that an classifier is \emph{strictly} more performant if the ROC curve is above a baseline for all efficiencies.

% section figure_of_merit (end)


\section{Deep Learning}

Since it's first usage by it's current name~\cite{hinton06}, Deep Learning has taken on many forms and seen success in a variety of fields that have traditionally utilized human-engineered features to create classifiers and apply out-of-the-box machine learning algorithms. In particular, the field of Computer Vision has changed drastically. Since the 2012 ILSVRC winning entry by Alex Krizhevsky and the University of Toronto group ~\cite{alexnet}, Deep Learning -- in particular Convolutional Neural Networks -- have taken over vision-based machine learning, consistently showing human and recently super-human levels performance on key baseline datasets. The increasingly widespread availability of GPUs and associated numerical frameworks has made the time intensive estimation procedures associated with deep neural networks more feasible, and has allowed the size of models for image tasks to grow exponentially. For example, the Google team's contribution to ILSVRC 2014 -- the GoogLeNet~\cite{googlenet} -- consisted of 22 layers of convolutional black boxes called ``Inception Units'', and set the benchmarks both for accuracy and speed of a model on such a large scale. 

As it relates to our work, do not investigate large network architectures, rather we focus on  understanding what information and higher level representations a convolutional neural network will learn in the context of High Energy Physics. We let our knowledge of physics guide our investigations into visualization, understanding, and demystification of deep representations for physics. We shed light inside the black-box of deep learning in the context of object identification in HEP.


\section{Network Architecture}

%An in-depth examination of architectures via trial and error is left  future 

We begin with the notion that the discretization procedure outlined in Section \ref{sec:simulation} produces $25\times 25$ ``energy-scale'' images in one channel -- a High Energy Physics analogue of a grayscale image. We note that the images we work with are \emph{sparse} -- roughly 12\% of pixels are active on average. Future work can build on efficient techniques for exploiting the sparse nature of these images -- i.e., memoized convolutions. However, since speed is not our driving force in this work, we utilized convolution implementations defined for dense inputs.

\subsubsection{Architectural Selection} % (fold)
\label{ssub:architectural_selection}
We utilize a very simple convolutional architecture for our studies, consisting of two sequential \texttt{[Conv + Max-Pool + Dropout]} units, followed by two fully connected, dense layers. Our architecture can be succinctly written as 

\begin{equation}
  \mathtt{[Dropout \rightarrow Conv \rightarrow ReLU \rightarrow MaxPool] * 2 \rightarrow [Dropout \rightarrow FC \rightarrow ReLU] * 2 \rightarrow Sigmoid}.
\end{equation}

After early experiments with the standard $3\times 3$ kernel size, we discovered no improvement over a more basic MaxOut \cite{maxout:goodfellow} feedforward network. After further investigation into larger convolutional kernel size, we discovered that larger-than-normal kernels work well on our application. Though not common in the Deep Learning community, we hypothesize that this larger kernel size is helpful when dealing with sparse structures in the input images. In Table~\ref{tab:kernelsize}, we show the optimal kernel size of $11\times11$ while considering the metric outlined in Section~\ref{sec:figure_of_merit}.

\begin{table}[h!]
  \centering
  \begin{tabular}{r|c}
    \bfseries Kernel size & \bfseries AUC \\ 
    \hline
    $(3 \times 3)$ Conv & 14.770 \\
    \hline
    $(4 \times 4)$ Conv & 12.452 \\
    \hline
    $(5 \times 5)$ Conv & 11.061 \\
    \hline
    $(7 \times 7)$ Conv & 13.308 \\
    \hline
    $(9 \times 9)$ Conv & 17.291 \\
    \hline
    $(11 \times 11)$ Conv & 20.286 \\
    \hline
    $(15 \times 15)$ Conv & 18.140 \\
  \end{tabular}
  \caption{First layer convolution size vs. performance}
  \label{tab:kernelsize}
\end{table}
% table

We follow up the first layer of convolutions with Rectified Linear Unit activations, then utilize $(2, 2)$ max-pooling to downsample. We then use $(4\times 4)$ convolutions in the second convolutional unit + $(2, 2)$ max-pooling, and connect to 64 units then one final output.
% subsubsection architectural_selection (end)

\subsubsection{Implementation and Training} % (fold)
\label{ssub:implementation_and_training}


Event generation and simulation was conducted on the SLAC \texttt{atlint} cluster. All Deep Learning experiments were conducted in Python with the Keras~\cite{Keras} Deep Learning library on the Stanford Institute for Computational and Mathematical Engineering GPU cluster, utilizing NVIDIA C2070 graphics cards. 

We used 30 million training and 30 million testing samples, and trained networks using both the Adam~\cite{DBLP:journals/corr/KingmaB14} algorithm and Stochastic Gradient Descent with Nesterov Momentum~\cite{Nesterov:1983wy}. We found that SGD+Nesterov outperformed Adam, and thus is used in all following facts and figures.

% subsubsection implementation_and_training (end)


\section{Studies} % (fold)
\label{sec:studies}

To begin understanding what a deep network can learn about jet topology, we choose a finite region of phase space, and standardize our comparisons. In an effort to define a standard way that physics object identification using machine learning should be conducted, we exactly define our procedure for comparisons. In particular, we restrict our studies to $250$ GeV $\leq p_T \leq 300$ GeV, and confine ourselves to a $65$ GeV $\leq m \leq 95$ GeV mass window, wholly containing the peak of the $W$. 

We construct a scaffolded and multi-approach series of methodologies for understanding, visualizing, and validating neural networks within HEP.

\subsection{Coarse Studies} % (fold)
\label{sub:coarse_studies}

To a first order, the first desirable characteristic is a simple performance improvement over the standard physics-driven variables for discrimination. In particular, we compare our network to $n$-subjettiness~\cite{nsub} and the jet mass. We henceforth refer to $n$-subjettiness as $\tau_{21}$ for our purposes, as $\tau_{2}/\tau_{1}$ is relevant for our classification problem.

In Figure~\ref{fig:combinedROC}, we illustrate the performance gains of a deep neural network over both $\tau_{21}$ and the 2D likelihood of $\tau_{21}$ and jet mass. 


\begin{figure}[!htbp]
  \centering
  \includegraphics[width=0.95\textwidth]{figures/combined-roc.pdf}
  \caption{Receiver Operating Characteristic (ROC) over coarse sample}
  \label{fig:combinedROC}
\end{figure}

We also provide a comparison to a 3D likelihood constructed on $\tau_{21}$, jet mass, and the deep network output itself. We can gain a significant piece of insight from this. Note how in Figure~\ref{fig:combinedROC} we can see that the DNN represents a large gain on a physics-only likelihood. However, when we explicitly include the physics variable in a 3D likelihood, we see a small but definitively non-zero performance gain. This implies that the performance boost \emph{by definition} is getting its gain from something that is not \emph{fully} encapsulated in $\tau_{21}$ and jet mass. 

Though important on it's own, this figure of merit does little to help drive understanding in the context of HEP. Such an increase begs further questions -- what is this gain, and where does it come from? Why is the DNN able to pick up on this?



\subsubsection{Understanding what is learned} % (fold)
\label{ssub:understanding_what_is_learned}

% subsubsection understanding_what_is_learned (end)
In Figure~\ref{fig:convkernels}, we first examine the $11\times11$ convolutional filters in the first layer and look for structure. In

In order to understand what we learn, we first take a look \emph{inside} the deep network. 
\begin{figure}[bt]
  \begin{center}
      \subfloat[$(11\times11)$ convolutional kernels from first layer \label{subfig:filters}]{
        \includegraphics[width=0.5\textwidth]{figures/conv-filts.pdf}
      }
      \subfloat[Convolved Jet Image differences\label{subfig:convolvedfilters}]{
        \includegraphics[width=0.5\textwidth]{figures/conv-diffs-global.pdf}
      }
      \caption{Convolutional Kernels (left), and convolved feature differences in jet images (right)}
      \label{fig:convkernels}

    \end{center}
\end{figure}



Blah... linear correlations with pixels





\subsubsection{Physics in Deep Representations} % (fold)
\label{ssub:physics_in_deep_representations}


\begin{figure}[!htbp]
  \centering
  \includegraphics[width=0.95\textwidth]{figures/pixel-activations-corr.pdf}
  \caption{Per-pixel linear correlation with DNN output}
  \label{fig:corr}
\end{figure}


\begin{figure}[bt]
  \begin{center}
      \subfloat[Sculpted QCD $\Delta R$ distribution\label{fig:sculpteddR}]
      {
        \includegraphics[width=0.5\textwidth]{figures/dR-dist-by-CNN.pdf}
      }
      \subfloat[Sculpted QCD $\tau_{21}$ distribution\label{fig:sculptednsj}]
      {
        \includegraphics[width=0.5\textwidth]{figures/tau-dist-by-CNN.pdf}
      }
      \\
      \subfloat[Sculpted QCD mass distribution\label{fig:sculptedmass}]
      {
        \includegraphics[width=0.5\textwidth]{figures/mass-dist-by-CNN.pdf}
      }

      \caption{Sculpted QCD distributions}
      \label{fig:qcdsculpt}

    \end{center}
\end{figure}





% subsubsection physics_in_deep_representations (end)

% subsection coarse_studies (end)

\subsection{Flat Hypercube Studies} % (fold)
\label{sub:flat_hypercube_studies}


Here, we see the ROC blah...

\begin{figure}[htbp]
  \centering
  \includegraphics[width=0.95\textwidth]{figures/roc-cube-inside.pdf}
  \caption{ROC Curve for weigth-flattened hypercube, with $m\in[65, 95]\mathsf{GeV}$,  $p_T\in[250, 300]\mathsf{GeV}$, and  $\tau_{21}\in[0.2, 0.8]$}
  \label{fig:rocCube}
\end{figure}

% subsection flat_hypercube_studies (end)



\subsection{Small Window Studies} % (fold)
\label{sub:small_window_studies}


\begin{figure}[bt]
  \begin{center}
  
      \subfloat[Average $W'\rightarrow WZ$ image \label{subfig:sig_window}]{
        \includegraphics[width=0.5\textwidth]{figures/avg-benwindow-sig.pdf}
      }
      \subfloat[Average QCD image \label{subfig:bkg_window}]{
        \includegraphics[width=0.5\textwidth]{figures/avg-benwindow-sig.pdf}
      } \\
      \subfloat[Average image difference \label{subfig:windowdiff}]{
        \includegraphics[width=0.5\textwidth]{figures/avg-benwindow-diff-clipped.pdf}
      }
      \caption{$W'\rightarrow WZ$ (left) and QCD (right) average jet-images, and Signal - Background image difference (bottom)
      \label{fig:meanImagesWindow} }
    \end{center}
\end{figure}  


Performance inside the window, use a Fisher Discriminant...

\begin{figure}[htbp]
  \centering
  \includegraphics[width=0.95\textwidth]{figures/augwindow-roc.pdf}
  \caption{Receiver Operating Characteristic (ROC) over window sample}
  \label{fig:rocWindow}
\end{figure}

\subsubsection{Understanding what we learn} % (fold)
\label{ssub:understanding_what_we_learn}

\begin{figure}[htbp]
  \centering
  \includegraphics[width=0.65\textwidth]{figures/fld-benwindow.pdf}
  \caption{caption}
  \label{fig:fldWindow}
\end{figure}

\begin{figure}[htbp]
  \centering
  \includegraphics[width=0.65\textwidth]{figures/pixel-activations-corr-benwindow.pdf}
  \caption{caption}
  \label{fig:corrWindow}
\end{figure}


\begin{figure}[htbp]
  \centering
  \includegraphics[width=0.95\textwidth]{figures/conv-diffs-ben-window.pdf}
  \caption{caption}
  \label{fig:convkernelsWindow}
\end{figure}
% subsubsection understanding_what_we_learn (end)

% subsection small_window_studies (end)





% section studies (end)


\section{Acknowledgements}

This work is supported by the US Department of Energy (DOE) Early Career Research Program and grant DE-AC02-76SF00515. BN is supported by the NSF Graduate Research Fellowship under Grant No. DGE-4747 and by the Stanford Graduate Fellowship.  SDSI?

% \nocite{*}



% \bibliography{myrefs}





\end{document}
