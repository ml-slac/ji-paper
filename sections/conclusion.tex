\section{Outlook and Conclusions}
\label{sec:conclusion}
Jet Images are a powerful paradigm for visualizing and classifying jets.  We have shown that when applied directly to jet images, deep neural networks are a powerful tool for identifying boosted hadronically decaying $W$ bosons from QCD multijet processes.  These advanced computer vision algorithms outperform several known and highly discriminating engineered physics-inspired features such as the jet mass and $n$-subjettiness, $\tau_{21}$.  Through a variety of studies, we have shown that some of these features are {\it learned} by the network.  However, despite detailed studies to preserve the jet mass, this important variable seems to not be fully captured by the neural network.  Understanding how to fully learn the jet mass is a goal of our future work.

In addition to the standard performance metrics, we develop several techniques for quantifying and visualizing the {\it unique information} learned by the network.  This is studied by removing the information from jet mass and $\tau_{21}$ through a re-weighting or redaction of the phase space.  In this way, we can evaluate the performance of the network beyond these standard features to quantify the unique information learned by the network.  In addition to quantifying the amount of additional discrimination achieved by the network, we also show how the new information can be visualized through Fisher jet-images, and through the deep correlation jet image which displays the network output correlation with each input pixel.  These visualizations are a powerful tool for understand what the network is learning - in this case, colorflow patterns suggest that at least part of the unique information comes from the octet versus singlet nature of $W$ bosons and gluon jets.  These visualizations may even be useful in the future for engineering other simple variables which may be able to match the performance of the neural network.

Jet-images combined with deep neural networks connect the field of particle physics and computer vision, and have the potential to significantly improve the new physics searches at the LHC.  Through further studies of such deep neural networks, not only can significant gains in information be made, but completely new physics features may be discovered.