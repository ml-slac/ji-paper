\section{Outlook and Conclusions}
\label{sec:conclusion}
Jet Images are a powerful paradigm for visualizing and classifying jets.  We have shown that when applied directly to jet images, deep neural networks are a powerful tool for identifying boosted hadronically decaying $W$ bosons from QCD multijet processes.  These advanced Computer Vision algorithms outperform several known and highly discriminating engineered physics-inspired features such as the jet mass and $n$-subjettiness, $\tau_{21}$.  Through a variety of studies, we have shown that some of these features are {\it learned} by the network.  However, despite detailed studies to preserve the jet mass, this important variable seems to not be fully captured by the neural networks studied in this article.  Understanding how to fully learn the jet mass is a goal of our future work.

In this paper, we propose several techniques for quantifying and visualizing the information learned by the DNNs, and connect these visualizations with physics properties.  This is studied by removing the information from jet mass and $\tau_{21}$ through a re-weighting or redaction of the phase space.  In this way, we can evaluate the performance of the network beyond these features to quantify the unique information learned by the network.  In addition to quantifying the amount of additional discrimination achieved by the network, we also show how the new information can be visualized through through the deep correlation jet image which displays the network output correlation with each input pixel.  These visualizations are a powerful tool for understanding what the network is learning - in this case, colorflow patterns suggest that at least part of the unique information comes from the octet versus singlet nature of $W$ bosons and gluon jets.  These visualizations may even be useful in the future for engineering other simple variables which may be able to match the performance of the neural network.  

This edition of the study of jet images has built a new link between particle physics and computer vision by using state of the art deep neural networks for classifying high-dimensional high energy physics data.  By processing the raw jet image pixels with these advanced techniques, we have shown that there is a great potential for jet classification.  Many analyses at the LHC use boosted hadronically decaying bosons as probes of physics beyond the Standard Model and the methods presented in this paper have important implications for improving the sensitivity of these analyses.  In addition to improving tagging capabilities, further studies with deep neural networks will help us discover new features to improve our understanding and improve upon existing features to fully capture the wealth of information inside jets.