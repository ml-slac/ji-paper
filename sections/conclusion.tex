\section{Outlook and Conclusions}
\label{sec:conclusion}
Through the use of deep neural networks applied to the jet-image representation of boosted jets, we provide a first look at the powerful discrimination and probe the information learned with cutting edge Computer Vision techniques.  We find that the deep neural networks outperform several known and highly discriminating engineered features inspired by physics.  We see that several of these features are learned by the network, though we have not yet been able to fully capture the information of the jet mass, the most discriminating of the features, within the network.  Understanding how to fully learn the jet mass is a goal of our future work.

Going beyond standard performance metrics, we develop several techniques for examining the information learned by the network.  That is, through weighting the distributions of physics variables to be the same in signal and background jet-images, and separately though restricting the phase space for evaluation, we remove the known physics-inspired discriminating information.  In this way, we can evaluate the performance of the network beyond such known physics information in order to determine how much new information is contributing to the network performance.  We then provide a view of  this new information though the use Fisher jet-images, and through the deep correlation jet image which displays the network output correlation with each input pixel.  These visualization provide a deep insight to the discriminating information contained within the networks.

Jet-images combined with deep neural networks connect the field of particle physics and computer vision, and have the potential to significantly improve the new physics searches at the LHC.  Through further studies of such deep neural networks, not only can significant gains in information be made, but completely new physics features may be discovered.