\section{Figure of Merit} % (fold)
\label{sec:figure_of_merit}

As is commonly done in High Energy Physics, we eschew the commonly chosen metric of basic accuracy in favor of the Receiver Operating characteristic. This is because we must examine the entire spectrum of trade-off between Type-I and Type-II error, as many applications in physics will choose different points along the trade-off curve. We use a slight modification of the traditional ROC. For any discriminating variable, let $c$ be a threshold on the likelihood ratio on that variable, and let $w$ be the vector of weights over the entire evaluation sample. We define the \emph{rejection} of such a threshold is defined as 
$$
    \rho(c) = \frac{1}{\text{FPR}(c, w)},
$$
where $\text{FPR}(c, w)$ is the weighted false positive rate for using $c$ as a threshold.

We define the \emph{efficiency} of $c$ as 
$$
    \varepsilon(c) = \text{TPR}(c, w),
$$
where $\text{TPR}(c, w)$ is the weighted false positive rate for using $c$ as a threshold. We then evaluate our algorithms using the area under the line generated by $\{(\varepsilon(c), \rho(c)) : \varepsilon(c)\in [0.2, 0.8]\}$. We say that an classifier is \emph{strictly} more performant if the ROC curve is above a baseline for all efficiencies.

% section figure_of_merit (end)